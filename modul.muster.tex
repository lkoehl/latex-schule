\RequirePackage{tikz}
\usetikzlibrary{patterns}

% Wozu das noch. Kann ich nicht überall gray schreiben?
\definecolor{muster}{named}{gray}


% liniert
\NewDocumentCommand \liniert {O{\linewidth} m O{0.8cm}} {%
  \begin{tikzpicture}
    \draw(0, 0); % Abstand zum oberen Rand wahren
    \foreach \n in {1,...,#2} % Linien zeichnen
    \draw[color=gray](0, -#3*\n )--(0.99*#1, -#3*\n );
  \end{tikzpicture}%
}


% kariert
\NewDocumentCommand \kariert {s O{0} m O{0.5cm} d<>} {%
  \begin{tikzpicture}[baseline=(O.north)]
    % Star option
    \IfBooleanTF{#1}{% Größe mit festen Maßen festlegen
      \ifstrequal{#2}{0}{%
        \pgfmathsetlengthmacro{\width}{\linewidth-\pgflinewidth}
      }{%
        \pgfmathsetlengthmacro{\width}{#2-\pgflinewidth}
      }
      \pgfmathsetlengthmacro{\height}{#3}
    }{% Größe als Anzahl Kästchen festlegen
      \ifnumequal{#2}{0}{%
        \pgfmathsetlengthmacro{\width}{
          floor((\linewidth-\pgflinewidth)/#4)*#4
        }
      }{%
        \pgfmathsetlengthmacro{\width}{#2*#4}
      }%
      \pgfmathsetlengthmacro{\height}{#3*#4}
    }
    \draw[color=muster,step=#4]
    (0,0) rectangle (\width,-1*\height)
    (0,0) grid (\width,-1*\height);
    \node (O) at (0, -0.5) {};
    \IfValueT{#5}{
      \node[fill=white,anchor= west] at (0.25,-1) {#5};
    }
  \end{tikzpicture}
}


% Millimeterpapier
\NewDocumentCommand \millimeter {s O{0} m} {%
  \begin{tikzpicture}[baseline=(O.north)]
    % Star option
    \IfBooleanTF{#1}{ % Größe mit festen Maßen festlegen
      \ifstrequal{#2}{0}
      {  \pgfmathsetlengthmacro{\width}{\linewidth}  }
      {  \pgfmathsetlengthmacro{\width}{#2}  }
      \pgfmathsetlengthmacro{\height}{-1*#3}
      %
      \pgfmathsetlengthmacro{\anza}{\width-\pgflinewidth}
      \pgfmathsetlengthmacro{\anzb}{\width-(0.5\pgflinewidth)}
      \pgfmathsetlengthmacro{\anzc}{\width-(0.1\pgflinewidth)}
    }{ %
      \ifnumequal{#2}{0}
      {  \pgfmathsetlengthmacro{\width}{\linewidth}  }
      {  \pgfmathsetlengthmacro{\width}{(#2+1)*1cm}  }
      \pgfmathsetlengthmacro{\height}{#3*1cm}
      %
      \pgfmathtruncatemacro{\anza}{(\width-\pgflinewidth)/1cm}
      \pgfmathtruncatemacro{\anzb}{(\width-(0.5\pgflinewidth))/1cm}
      \pgfmathtruncatemacro{\anzc}{(\width-(0.1\pgflinewidth))/1cm}
    }
    %
    \draw[color=muster!50,step=1mm,very thin]
    (0,0) grid (\anzc,\height);
    \draw[color=muster,step=5mm,thin]
    (0,0) grid (\anzb,\height);
    \draw[color=muster,step=10mm,thick]
    (0,0) grid (\anza,\height);
  \end{tikzpicture}
}


% punktiert
\pgfdeclarepatternformonly
{dotgrid} 				% name
{\pgfpoint{0pt}{0pt}}     % lower left
{\pgfpoint{5mm}{5mm}}		% upper right
{\pgfpoint{5mm}{5mm}}		% tile size
{							% shape description
  \pgfpathcircle{\pgfqpoint{2.5mm}{2.5mm}}{.7pt}
  \pgfusepath{fill}
}
\NewDocumentCommand \punktiert {s O{0} m} {%
  \begin{tikzpicture}[baseline=(O.north)]
    % Star option
    \IfBooleanTF{#1}{% Größe mit festen Maßen festlegen
      \ifstrequal{#2}{0}{%
        \pgfmathsetlengthmacro{\width}{\linewidth}
      }{%
        \pgfmathsetlengthmacro{\width}{#2}
      }
      \pgfmathsetlengthmacro{\height}{#3}
    }{% Größe in Zentimetern
      \ifnumequal{#2}{0}{%
        \pgfmathsetlengthmacro{\width}{\linewidth}
      }{%
        \pgfmathsetlengthmacro{\width}{#2*5mm}
      }%
      \pgfmathsetlengthmacro{\height}{#3*5mm}
    }
    \fill [pattern=dotgrid,pattern color=muster] (0,0) rectangle (\width, \height);
  \end{tikzpicture}
}